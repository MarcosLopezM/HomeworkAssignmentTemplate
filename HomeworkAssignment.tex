\documentclass[12pt]{article}
\usepackage[export]{adjustbox}
\usepackage{amsfonts, amsmath, amssymb, amsthm}
\usepackage[english, spanish]{babel}
\usepackage{bm}
\usepackage{enumitem}
\usepackage{fancyhdr}
\usepackage{geometry}
\usepackage{graphicx}
\usepackage{kantlipsum}
\usepackage{lastpage}
\usepackage{mathtools}
\usepackage{microtype}
\usepackage{physics}
\usepackage[nolabel]{showlabels}
\usepackage{siunitx}
\usepackage{titling}
\usepackage{xcolor}
\usepackage{xparse}
\usepackage{xspace}
\usepackage{hyperref}
\usepackage{cleveref}

%%%%%%%%%%%%%%%%%%%%%%%%%%%%%%%%%%%%%%%%%%%%%%%%%%%%%%%%%%
%%%%%%%%%%%%%%%%%%%%%%%%%%%%%%%%%%%%%%%%%%%%%%%%%%%%%%%%%%
%%%%%%%%%%%%%%%%%%Configuraciones extras%%%%%%%%%%%%%%%%%%
\definecolor{base3}{RGB}{253, 246, 227}%
\definecolor{pinkwave}{RGB}{255, 0, 128}%
\pagecolor{base3}
\graphicspath{{img/}}
\setlength{\parindent}{2em} % Sangría
\setlength{\parskip}{0.5em} % Espacio entre párrafos
\linespread{1.1} % line spacing
\setlength{\jot}{10pt} % Space between lines in multiline eqs

% Configuración de ambiente para problema
\theoremstyle{definition}
\newtheorem{problema}{Problema}

% Configuración del paquete hyperref
\hypersetup{
    colorlinks = true,%
    linkcolor={[rgb]{0,0.2,0.6}},%
    citecolor={[rgb]{0,0.6,0.2}},%
    filecolor={[rgb]{0.8,0,0.8}},%
    urlcolor={[rgb]{0.8,0,0.8}},%
    runcolor={[rgb]{0.8,0,0.8}},% 
    menucolor={[rgb]{0,0.2,0.6}},%
    linkbordercolor={[rgb]{0,0.2,0.6}},%
    citebordercolor={[rgb]{0,0.6,0.2}},%
    filebordercolor={[rgb]{0.8,0,0.8}},%
    urlbordercolor={[rgb]{0.8,0,0.8}},%
    runbordercolor={[rgb]{0.8,0,0.8}},%
    menubordercolor={[rgb]{0,0.2,0.6}},% 
    pdftitle={Tarea X},%
    pdfauthor={López Merino Marcos},%
    pdfsubject={Subject},%
    pdfkeywords={Facultad de Ciencias, UNAM, materia, palabras clave},%
    unicode = true%
}

% Configuración del paquete siunitx
\sisetup{
	output-decimal-marker = {.}, 
	per-mode = symbol-or-fraction,
	separate-uncertainty = false,
	exponent-product = \cross,
    % inter-unit-product = \ensuremath{{}\vdot{}}
}

% Geometría del documento
\geometry{
    letterpaper,
    top = 0.6in,
    bottom = 0.8in,
    left = 0.8in,
    right = 0.8in
}

%%%%%%%%%%%%%%%%%%Nuevos comandos%%%%%%%%%%%%%%%%%%
\newcommand*{\group}{XXXX}
\newcommand*{\classname}{Subject}
\newcommand*{\homeworknumber}{Tarea X}
\newcommand*{\subject}{Tema}
\newcommand*{\teacher}{Dr. Name LastName LastName}

% unit vector i
\newcommand*{\uveci}{{\bm{\hat{\textnormal{\bfseries\i}}}}}
% unit vector j
\newcommand*{\uvecj}{{\bm{\hat{\textnormal{\bfseries\j}}}}}
% unit vector
\DeclareRobustCommand{\uvec}[1]{{%
  \ifcsname uvec#1\endcsname
     \csname uvec#1\endcsname
   \else
    \bm{\hat{\mathbf{#1}}}%
   \fi
}}% 
\newcommand{\idest}{\emph{i.e.},} % id est
% Espacio vectorial, e.g., ℝ, ℂ, ℕ, etc.
\NewDocumentCommand{\vecspace}{m o}{%
  \IfValueTF{#2}{%
    \mathbb{#1}^{#2}%
  }{%
    \mathbb{#1}%
  }%
}
\newcommand*{\e}{\mathrm{e}} % exponential
\newcommand{\solution}{\vspace*{4pt} \textbf{Solución}\vspace*{4pt}}

%%%%%%%%%%%%%%%%%%Portada y configuración%%%%%%%%%%%%%%%%%%
% Configuración de portada
\setlength{\droptitle}{-80pt} % raise the title

% Portada
\title{
    % \vspace{-1.5cm}
    \textbf{\homeworknumber: \subject}\\
    \normalsize\vspace{0.1in}\small{\textbf{Entrega}:~\today}\\
    \vspace{0.1in}\large{Prof.: \teacher}
    \vspace{-0.5cm}
}
\author{
    \textbf{Alumno:}\\ 
    López Merino, Marcos
}
\date{}

%%%%%%%%%%%%%%%%%%Header and footer%%%%%%%%%%%%%%%%%%
\setlength{\headheight}{15.2pt}
\pagestyle{fancy}
\lhead{Grupo \group, Sem. 2022-2}
\chead{\classname}
\rhead{\homeworknumber}
\lfoot{\includegraphics[scale = 0.2, valign = c]{LogoFCUNAMcolor.pdf}}
\cfoot{\textbf{Facultad de Ciencias, UNAM}}
\rfoot{Pág. \thepage \hspace{1pt} de \pageref{LastPage}}

\renewcommand{\headrulewidth}{0.5pt}
\renewcommand{\footrulewidth}{0.5pt}
%%%%%%%%%%%%%%%%%%%%%%%%%%%%%%%%%%%%%%%%%%%%%%%%%%%%%%%%%%
%%%%%%%%%%%%%%%%%%%%%%%%%%%%%%%%%%%%%%%%%%%%%%%%%%%%%%%%%%

\begin{document}
    \maketitle
    \thispagestyle{fancy}
    \begin{problema}
    \kant[1]
\end{problema}
\solution

\kant[1-2]

\begin{align}
    \dv{t}(\pdv{\mathcal{L}^{\prime}}{\dot{q}_{j}}) - \pdv{\mathcal{L}^{\prime}}{q_{j}} &= 0,\\
    \dv{t}(\pdv{\dot{q}_{j}}(\mathcal{L} + \dv{F(q_{i},t)}{t})) - \pdv{q_{j}}(\mathcal{L} + \dv{F(q_{i},t)}{t}) &= 0,\\
    \underbrace{\left[\dv{t}(\pdv{\mathcal{L}^{\prime}}{\dot{q}_{j}}) - \pdv{\mathcal{L}^{\prime}}{q_{j}}\right]}_{\text{\textcolor{blue}{(A)}}} + \underbrace{\left[\dv{t}(\pdv{\dot{q_{j}}}(\dv{F(q_{i},t)}{t})) - \pdv{q_{j}}(\dv{F(q_{i},t)}{t})\right]}_{\text{\textcolor{red}{(B)}}} &= 0.
\end{align}
    \begin{problema}
    \kant[1]
\end{problema}
\solution

\kant[1-3]

\begin{align*}
    m^{2}\ddot{x}\dot{x}^{2} + 2m\ddot{x}V(x) + 2m\dot{x}^{2}V^{\prime}(x) - m\dot{x}^{2}V^{\prime}(x) + 2V(x)V^{\prime}(x) &= 0,\\
    m^{2}\ddot{x}\dot{x}^{2} + 2m\ddot{x}V(x) + m\dot{x}^{2}V^{\prime}(x) + 2V(x)V^{\prime}(x) &= 0,\\
    \dfrac{1}{2}\left[(m\dot{x}^{2} + 2V(x))m\ddot{x} + (m\dot{x}^{2} + 2V(x))V^{\prime}(x)\right] &= 0,\\
    \left(\dfrac{1}{2}m\dot{x}^{2} + V(x)\right)m\ddot{x} + \left(\dfrac{1}{2}m\dot{x}^{2} + V(x)\right)V^{\prime}(x) &= 0,\\
    \textcolor{blue}{\left(\dfrac{1}{2}m\dot{x}^{2} + V(x)\right)}\textcolor{red}{(m\ddot{x} + V^{\prime}(x))} &= 0.
\end{align*}
    % \begin{problema}
    \kant[1]
\end{problema}
\solution

\kant[1-2]

\begin{align}
    \dv{t}(\pdv{\mathcal{L}^{\prime}}{\dot{q}_{j}}) - \pdv{\mathcal{L}^{\prime}}{q_{j}} &= 0,\\
    \dv{t}(\pdv{\dot{q}_{j}}(\mathcal{L} + \dv{F(q_{i},t)}{t})) - \pdv{q_{j}}(\mathcal{L} + \dv{F(q_{i},t)}{t}) &= 0,\\
    \underbrace{\left[\dv{t}(\pdv{\mathcal{L}^{\prime}}{\dot{q}_{j}}) - \pdv{\mathcal{L}^{\prime}}{q_{j}}\right]}_{\text{\textcolor{blue}{(A)}}} + \underbrace{\left[\dv{t}(\pdv{\dot{q_{j}}}(\dv{F(q_{i},t)}{t})) - \pdv{q_{j}}(\dv{F(q_{i},t)}{t})\right]}_{\text{\textcolor{red}{(B)}}} &= 0.
\end{align}
    % \begin{problema}
    \kant[1]
\end{problema}
\solution

\kant[1-3]

\begin{align*}
    m^{2}\ddot{x}\dot{x}^{2} + 2m\ddot{x}V(x) + 2m\dot{x}^{2}V^{\prime}(x) - m\dot{x}^{2}V^{\prime}(x) + 2V(x)V^{\prime}(x) &= 0,\\
    m^{2}\ddot{x}\dot{x}^{2} + 2m\ddot{x}V(x) + m\dot{x}^{2}V^{\prime}(x) + 2V(x)V^{\prime}(x) &= 0,\\
    \dfrac{1}{2}\left[(m\dot{x}^{2} + 2V(x))m\ddot{x} + (m\dot{x}^{2} + 2V(x))V^{\prime}(x)\right] &= 0,\\
    \left(\dfrac{1}{2}m\dot{x}^{2} + V(x)\right)m\ddot{x} + \left(\dfrac{1}{2}m\dot{x}^{2} + V(x)\right)V^{\prime}(x) &= 0,\\
    \textcolor{blue}{\left(\dfrac{1}{2}m\dot{x}^{2} + V(x)\right)}\textcolor{red}{(m\ddot{x} + V^{\prime}(x))} &= 0.
\end{align*}
\end{document}